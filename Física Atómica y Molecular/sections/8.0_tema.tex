\begin{align*}
&\underline{\text{Simetrías en moléculas diatómicas:}}\\
&\centerdot \text{Generales: asociadas al nº cuántico } \Lambda = |M_L|\; \; 
\setlength{\fboxsep}{1pt}
\boxed{
\left.\hspace{-2pt}
\begin{array}{l}
\;\;\;\Lambda \;\;\;\;\;\,\;\;\;\;\;\;0\; \;\; 1\;\;  \; 2\; \; \; 3\;\; \; 4\; \; \;5\;\;\;6\ \ 7\;  ...  \\
\text{Capa}(\mathcal{O})\:\,\;\;\Sigma\; \;\, \Pi\;\; \Delta \;\; \Phi\; \;\, \Gamma \; \:\:H\;\;I\;\;K\; ... \!\! 
\end{array}
\right.
}\\
&\cdot \text{Sim. cilíndrica alrededor de eje internuclear $\equiv$ eje que une a los núcleos (eje Z)}\  
\\&\cdot \text{Sim. reflexión en el plano que contiene al eje internuclear. Transformación asociada: } \sigma_{\nu} \\
&\Lambda \neq 0 : \text{deg}=2: \pm \Lambda \;\scalebox{0.8}{\text{(no explicitar)}} \ \ \ \ \Lambda =0  \rightarrow \Sigma:\text{NO deg. }\sigma_\nu \Sigma^+ =\Sigma^+\;\;\; \sigma_\nu \Sigma^- = - \Sigma ^- 
\\
&\centerdot \text{Homonucleares (núcleo $A$ igual al núcleo $B$)}\\
&\cdot \text{Inversión respecto el origen O(0,0). Transformación asociada: $I$}\\
&\text{gerade (par): }I\mathcal{O}_g=\mathcal{O}_g\;\;\;\;\;\; \text{ungerade (impar): } I\mathcal{O}_u=-\mathcal{O}_u  
\\
&\cdot \text{Reflexión en plano $\perp$ eje internuclear. Transformación asociada: }R_z= C_2 \cdot I\\
&R_z \mathcal{O}=\mathcal{O}\equiv\text{enlazantes}\;\;\;\;\;R_z\mathcal{O}^*=-\mathcal{O}^*\equiv\text{antienlazantes ($g$ con $\Lambda$ impar)} 
\\
    &\underline{\text{Orbitales moleculares (O.M.)}}:
    \text{sigue Modelo e$^-$ Libres (MeL) }\rightarrow U_{ee}\approx 0
    \\&\phi_q^{(0)}(\vec{r}_i. \vec{R}^*)\!=\!\Pi_j^N \phi_j^e(\vec{r}_j, \vec{R}^*) \;\;\;\;      E_q^{(0)} \!=\! \sum_j\epsilon_j(\vec{R}^*)\;\;\;\;\lambda = |m_l|\; \; 
\left.\hspace{-2pt}
\begin{array}{l}
\;\;\;\lambda \;\;\;\;\;\;0\; \;\; 1\;\;  \; 2\; \; \; 3\;\; \; 4\; \; \;  ...  \\
\text{Capa}\:\,\;\;\sigma\; \;\, \pi\;\;\, \delta \;\; \,\varphi\; \;\; \gamma \; \; ... \!\! 
\end{array}
\right.
\\[-3pt]&
\text{e$^-$ introducidos de 2 en 2 en O.M. $\sigma$, de 4 en 4 en el resto de O.M.}\\
&\underline{\text{Términos moleculares (T.M.)}}: \text{consideramos }U_{ee}. \text{ Se denotan }\;^{2S+1}\Lambda, \text{ multip.: $2S+1$}\\
&\text{O.M. $\sigma$: tienen carácter $+\rightarrow $ solo dan lugar a T.M. $\Sigma^+$}\;\;\;\;\text{Resto O.M. dan lugar a $\Sigma^+, \Sigma^-$}\\
&\text{Est. fund. molécula denotado con $X$ delante del término (gral: término máx sim. y spin 0)}\\
&\underline{\text{Niveles orbitales}}: \text{consideramos acoplamiento S-O. Se denotan } ^{2S+1} \Lambda _{\Omega_z}\\
&\text{Proy. mom. ang. electrónico tot.}\equiv \Omega_z = (\Lambda -\Sigma^f), ..., (\Lambda +\Sigma^f); \text{ con $\Lambda$ fijo},\; \; \Sigma^{f}=M_S.\\
&\text{Desplazamiento de energía: }\Delta E=\mathcal{A}\Lambda\Sigma^f, \; \mathcal{A}\equiv\text{parámetro de acoplamiento spin-órbita}\\
&\underline{\text{Obtención términos y niveles moleculares:}} \;\;\;\; \text{\;\;\; $T\equiv$ términos de la forma $^{2S+1}\Lambda$}\\
&M_L= \Sigma_i m_{l_i}\;\;\;\; \lambda = k \rightarrow \pm m_l = \pm k\;\;\;\;\; \text{deg O.M.: }\prod_{i}^N \binom{\S_i}{\chi_i}\;\;
\begin{array}{l}
     \S_i= 2\text{ if $\sigma$, \:4 if other O.M.} \\
     \chi_i\equiv \text{nº e$^-$ en orbital}
\end{array}
\\
&S=s_i \oplus s_k \;\;\;\; \deg\text{T.M.: }\wp\cdot(2S+1),\;\;\; \; \wp=1\text{ if $\Sigma$, 2 if other T.M.}
\\ 
&\centerdot \text{e$^-$ no equivalentes (i.e. O.M. sin superíndices) \;\;\;\;\; NO DISCRIMINAR}
\end{align*}
\begin{center}
\renewcommand{\arraystretch}{1.5} % aumenta la altura de las filas
\begin{tabular}{|c|c|c|c|}
\hline
\textbf{ } & \multicolumn{2}{c|}{$T_a$ $(\Lambda_a,\,S_a)$} & \textbf{Términos resultantes} \\ 
\hline
\multirow{2}{*}{$T_i$ $(\Lambda_i,\,S_i)$} 
 & $\Lambda=\Lambda_i+\Lambda_a$ & $S=S_i \oplus S_a$ & $T_{\mathfrak{a}}, T_{\mathfrak{b}}, ...$ \\ 
\cline{2-4}
 & $\Lambda=|\Lambda_i-\Lambda_a|$ & $S=S_i \oplus S_a$ & $T_{\mathfrak{c}}, T_{\mathfrak{d}}, ...$ \\ 
\hline
\end{tabular}
\end{center}
\begin{align*}
    &\text{If > 2 e$^-$, acople 2, resultado en sucesivas filas $T_i$ y se acopla con otro, así iterativamente}\\
    &\text{If $\Lambda=0$, proveniente de $T= \Sigma^{+ /-}, T$ resultante: $\Sigma^{{\text{prod. \scalebox{0.6}{$+/-$}}}}$  \; \; else: $T=\Sigma^+, \Sigma^-$}\\
    &\text{If }\Lambda_i = 0 \text{ y/o } \Lambda_a=0\;\; \text{ejecuta algoritmo 1! vez.}\\
    &\centerdot\text{e$^-$ equivalentes (i.e. O.M. con superíndices): trata e$^-$ no equiv, discrimina AL FINAL}\\ &\text{Paridad fun. onda espacial = $-$ Paridad fun. onda spin}\;\;
   \begin{array}{c}
    \cdot \text{espacial según }\Lambda\,   (\Sigma\text{ según }\pm )  \\
    \cdot \uparrow \downarrow \equiv \text{impar},\;\;\uparrow\uparrow, \downarrow\downarrow\equiv \text{par}         
   \end{array} 
\\
&\text{If > 2 e$^-$, discrimina,, no repeat nº cuánt., e.g. }(\lambda \uparrow, \lambda \downarrow, -\lambda \uparrow)\; \;\; \text{Trick: }\lambda^k = \lambda^{\S-k}
\\
&\centerdot\text{configuración mixta (i.e. equiv y no equiv): acoplar por separado, después componer}\\
&\text{If orbital completo $\rightarrow$ ignorar (no contribuye, tienen: $\Lambda=0, S=0 \Rightarrow ^1\!\!\Sigma^+\equiv\overset{\scalebox{0.9}{\text{actúa as}}}{\text{identidad}})$}\\
&\text{Configuración electrónica fundamental (moléculas homonucleares, hasta Ne$_2$): }\\
&\!\!\!(\sigma_g1s)^2
(\sigma_u^*1s)^2 
(\sigma_g2s)^2 
(\sigma_u^*2s)^2
\underline{(\sigma_g2p)^2
(\pi_u2p)^4}
(\pi_g^*2p)^4
(\sigma_u^* 2p)^2
\;
\text{ if $Z\!\leq\!7$, swap subrayado}
\\
&\text{La paridad $g$ o $u$ de un T.M. dada por producto de paridad $g,u$ de los O.M. de partida}\\
&\text{Orden de enlace: }\tfrac{1}{2}[\#e^- \text{ enlazantes} - \#e^- \text{ antienlazantes}] \;\;\; \overset{\uparrow \text{O.E.} \uparrow \text{estable}}{\text{O.E.}<0 \rightarrow\text{moléc. disocia}}\\
&\text{En moléculas heteronucleares (distintos núcleos) $\rightarrow$ tratamiento similar a homonucleares,}\\
&\text{pero $\centerdot$ desaparece $g,u \;(\nexists $ inv. resp. O)\;\;\; $\centerdot $ ordenamiento específico según molécula}
\end{align*}
