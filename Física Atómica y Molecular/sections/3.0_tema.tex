\section*{Átomos con 2 e$^-$}
\begin{align*}
    &\hat{\mathcal{H}}=-\frac{\hbar^2}{2m}\nabla_{r_1}^2-\frac{Ze^2}{4\pi\varepsilon_0r_1}-\frac{\hbar^2}{2m}\nabla_{r_2}^2-\frac{Ze^2}{4\pi\varepsilon_0r_2}+H_{e-e}\; \; \;\;  H_{e-e} \equiv H'= \frac{e^2}{4\pi\varepsilon_0r_{12}}\; \;\;\\
&\Psi(q_1,q_2)=\psi(\vec{r_1,}\vec{r_2})\chi(s_1,s_2)\; \;\;\;\; \chi_{\frac{1}{2},+\frac{1}{2}}(i) = \alpha(i) \; \;\;\;\; \chi_{\frac{1}{2},-\frac{1}{2}}(i)=\beta(i)\\
&\left.\hspace{-5pt}
\begin{array}{l}
\chi_1 = \alpha(1)\alpha(2) \;\;(\uparrow\uparrow)\\
\chi_2 = \alpha(1)\beta(2)\;\; (\uparrow\downarrow)\\
\chi_3 = \beta(1)\alpha(2) \;\;(\downarrow\uparrow)\\
\chi_4 = \beta(1)\beta(2)\;\; (\downarrow\downarrow)\\
\end{array}
\right.\;\;\;
\left.\hspace{-5pt}
\begin{array}{c}
\text{$\chi_{2,3}$ no son autofuns de $\hat{\vec{S}}$}\\
\chi_+ = \frac{1}{\sqrt{2}} [\chi_2 + \chi_3]\\
\chi_- = \frac{1}{\sqrt{2}} [\chi_2 - \chi_3]\\
\text{$\chi_{\pm}$ sí son autofuns de $\hat{\vec{S}}$}
\end{array}
\right.
\;\;
\left.\hspace{-6pt}
\begin{array}{l}
\chi_1\,\equiv \chi_{1,1}\\
\chi_+\equiv \chi_{1,0}\\ 
\chi_4\,\equiv \chi_{1,\scalebox{0.9}{-}1}\\
\chi_-\equiv \chi_{0,0}\\
\end{array}
\right.
\left.\hspace{-10pt}
\begin{array}{l}
\rule{0.15pt}{0.7cm}\raisebox{7pt}{\text{ Triplete (S=1,\! SIM)}}\\
\rule{0.15pt}{0.2cm} \text{ Singlete (S=0,\! ANTI)}
\end{array}
\right.\\
&\psi(\vec{r_1},\vec{r_2})=\psi(\vec{r_1})\psi(\vec{r_2})\;\;\;\psi(\vec{r_1},\vec{r_2}) = \pm \psi(\vec{r_2},\vec{r_1})\;\;
\left.\hspace{-9pt}
\begin{array}{l}
+ \; \text{  SIM (estado para)}\\
- \; \text{ANTI (estado orto)}
\end{array}
\right.\\
&\underline{\text{Modelo de partículas independientes (M.P.I)}} \text{ Desprecio $H_{e-e}$. (ídem áts. multi e$^{-}$)}\\
&\psi_{\pm}^{(0)}(\vec{r_1}, \vec{r}_2) = \frac{1}{\sqrt{2}}\left[\psi_a(\vec{r}_1)\psi_b(\vec{r}_2) \pm \psi_b(\vec{r}_1)\psi_a(\vec{r}_2)\right]\; \; \; \psi_k = \psi_{n_i, l_i, m_i}\;\; k=a,b=1,2\\
&\text{Energía: } E_{n_{1,2}}^{(0)} = E_{n_1} + E_{n_2} = -Ry\;  Z^2 \left(\frac{1}{n_1^2}+\frac{1}{n_2^2}\right) \; \; \
\left.\hspace{-9pt}
\begin{array}{c}
\text{Para 1s: }\psi_- = 0,\vspace{3pt} \\
\psi_+\! =\! \frac{1}{\pi}\!\left(\frac{Z}{a_0}\right)^3 \!\exp\left[\frac{-Z(r_1+r_2)}{a_0}\!\right]
\end{array}
\right.
\end{align*}
\begin{align*}
&\underline{\text{Aprox. Campo Central (C.C.)}} \text{ (potenciales centrales, $H_{e-e}\approx 0$)}\\
&\hat{U}^i_c (r_i)=-\frac{(Z-S(r_i))e^2}{4\pi\varepsilon_0 r_i}\; \; \; Z_{ef}^i (r_i)=Z-S(r_i)\; \; \;
\left.\hspace{-9pt}
\begin{array}{c}
S(r_i)\equiv \text{Parámetro apantallamiento} \\
r\rightarrow 0\Rightarrow S\rightarrow 0 \;\;\;\;\;
r \rightarrow \infty \Rightarrow S \rightarrow 1
\end{array}
\right.\\
&\text{Si $S\equiv$ const.: mismo resultado M.P.I. pero } Z \mapsto Z_{ef}\;\;\;
\left.\hspace{-2pt}
\begin{array}{l}
E_{1s,nl}^{(0)} =E_{1s}+E_{nl}\\
n\equiv \text{ fijo: } \uparrow l \Rightarrow \uparrow E_{nl}
\end{array}
\right.\\ 
&\text{REPULSIÓN ELECTRÓNICA (incluyo $H_{e\scalebox{0.9}{-}e}$): $n,l\equiv$ fijos: $E_{\text{triplet}}<E_{\text{singlet}}$}\\
&\vec{l}_1, \vec{l}_2, \vec{s}_1, \vec{s}_2 \text{ no conmutan con } H,\; \; \; \vec{L}=\vec{l}_1+\vec{l}_2, \vec{S}=\vec{s}_1+\vec{s}_2 \text{ sí conmutan con } H\\
&\text{Términos espectrales: } n^{2S+1} L\; \; 
\setlength{\fboxsep}{1pt}
\boxed{
\left.\hspace{-2pt}
\begin{array}{l}
0\; \;\; 1\;\;  \; 2\; \; \; 3\;\; \; 4\; \; \; 5\; ...  \!\!\\
S\; \; P\; \; D\; \; F\; \; G\; \; H \; ... \!\! 
\end{array}
\right.
}
\; \; \text{(letra donde pone $L$)}\\
&\underline{\text{Método perturbativo}} \;\; E_0^{(1)}=E_0^{(0)}+ \Delta E^{(1)}_0 \; \; \; 
\text{para $1s^2:$ }\Delta E^{(1)}_0=\langle H'\rangle_{\psi_0^{(0)}}= E_0^{(0)} \frac{5}{8}\frac{1}{Z}\\
&\underline{\text{Método variacional}} \; \; \phi = \psi_{1\text{s}}^{Z_{ef}}(r_1)\cdot\psi_{1\text{s}}^{Z_{ef}}(r_2)\; \; \; E[\phi]= \frac{\langle \phi |\mathcal{\hat{H}}|\phi \rangle}{\langle\phi|\phi\rangle}\; \;\;\;\; \frac{\partial E[\phi]}{\partial Z_{ef}}=0 \Rightarrow\\
\Rightarrow &\text{ para } 1s^2: Z_{ef}^*=Z-\frac{5}{16} \Rightarrow E_0^{(var)}=2Ry\left(Z_{ef}^*\right)^2-2ZZ_{ef}^*+ \frac{5}{8}Z_{ef}^*\\
&\underline{\text{Configuraciones Excitadas}}\: \:1s nl: \; E^{(0)}_{1s,nl}=E_{1s}+E_{n}\; \; \; \; \Delta E_{\pm}^{(1)}= J\pm K \; \; (\text{según }\psi_{\pm})\\
&J_{nl} = 2 Ry\int_0^{\infty} dr_2\, r_2^2 \, [R_{nl}(r_2)]^2 
          \int_0^{\infty} dr_1\, r_1^2 \, [R_{10}(r_1)]^2 \, \frac{1}{r_>}\; \; 
\left.\hspace{-2pt}
\begin{array}{l}
r_<=\min(r_1,r_2)\\ 
r_>=\max(r_1,r_2)
\end{array}
\right.\\
&K_{nl} = \frac{2 Ry}{2l + 1} 
          \int_0^{\infty} dr_2\, [r_2^2 R_{10}(r_2) R_{nl}(r_2)] 
          \int_0^{\infty} dr_1\, [r_1^2 R_{10}(r_1) R_{nl}(r_1)]
          \frac{r_<^{\,l}}{r_>^{\,l+1}}\\
&\underline{\text{Principio de Exclusión Pauli}:} \;\;\mathcal{P}_{ij}\Psi_{ij} =\pm \Psi_{ji} \; \; \; \;  \left.\hspace{-2pt}
\begin{array}{c}
+ \equiv \text{Bosones}\\
\text{spin entero}
\end{array}
\right.
 \; \; \; \;
 \left.\hspace{-8pt}
\begin{array}{c}
- \equiv \text{Fermiones}\\
\text{spin semi-entero}
\end{array}
\right.
\end{align*}

