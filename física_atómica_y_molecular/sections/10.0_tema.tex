\section*{Espectroscopía molecular\normalfont{\text{ (aprox. D.E.: $\Delta \Lambda =0,\pm 1, \;\Delta S = 0, \;  \Sigma^+\!\nleftrightarrow \Sigma^-,\; \text{homo: } u\leftrightarrow g$)}}}
\begin{align*}
    &\underline{\text{Tipos de espectros}}:
    \\
    &\centerdot \text{Rotacionales puros ($\Delta \upsilon =0,$ no cambia esp. electrónico, moléculas NO lineal-simétricas)}\\
    &\Delta J=\pm 1 \! \begin{array}{c}
         \text{absorción}  \\[-2pt]
          \text{emisión}
    \end{array}
    \; \Delta E_{J\rightarrow J+1, \upsilon}^{(0)}=2 B( J+1) \;\;\; 
    \begin{array}{c}
         \text{serie líneas equiespaciadas } 2 B  \\
          \text{dist. internuc.} \equiv R_e:\; I_0=\mu R_e^2 \;\;\;
    \end{array}
    \\
    &\text{Corrección centrífuga }\Delta E_{J\rightarrow J+1, \upsilon}^{(1)}=2 B( J+1)-4D_e (J+1)^3 \\
    &\centerdot \text{Vibaracionales puros ($J =0$, no cambia esp. electrónica)}\\&
    \tfrac{\Delta E_{\upsilon\rightarrow\upsilon'}}{hc}=\tfrac{\nu_0}{c}(\upsilon' - \upsilon)- \chi_e\tfrac{\nu_0}{c}(\upsilon'^2 +\upsilon'-\upsilon^2-\upsilon) \;\;\;\;\; \text{Deducir de $E_{J,\upsilon}^{(2)}/hc$}
    \\
    &\centerdot \text{Vibración-rotación ($\nexists$ vib. puros, no cambia esp. electrónico, aprox rotor rígido y O.A.)}\\
    &\Delta \upsilon =\pm 1 \begin{array}{c}
         \text{absorción}  \\[-2pt]
          \text{emisión}
    \end{array}\;
    \Delta J=\pm 1 \;\;\;(\Delta J =0,\pm 1 \text{ if } \Lambda \neq 0)\;\;\; \text{serie líneas equiespaciadas } 2 B\\
    &\Delta J= +1 \;\;\; \Delta E_{J\rightarrow J+1, \upsilon\rightarrow \upsilon+1}=\hbar \omega_0 + 2B(J+1) \equiv \text{Rama R}\\
    &\Delta J= -1 \;\;\; \Delta E_{J\rightarrow J-1, \upsilon\rightarrow \upsilon+1}=\hbar \omega_0 - 2B(J+1) \equiv \text{Rama P}\\
    &\Delta J= \;\;0 \;\;\;(\text{frecuencia fundamental } \nu_0 \Leftrightarrow \Lambda, \Lambda' \neq 0) \equiv \text{Rama Q}\\
    &\text{Transiciones se denotan con $P(J), R(J),... \;\;\;$ con $J$ de partida}\\
    &\text{En aprox. anarmónica: } \Delta \upsilon = \pm 1, \underline{\pm 2, \pm 3, ...} \;\;\text{subrayado: sobretonos (1º sobretono: $0\rightarrow2$)}
    \\
    &\centerdot \text{Electrónicos (se desprecia estructura rotacional)}\\
    &\text{Desplazamiento Stokes: $\Delta E_{\text{Stokes}} = E_{\text{abs.}}\!\!-\!\! E_{\text{emis.}} \!=(2S-1)\hbar\omega_0\;\;\;\; S= \frac{\mu \omega_0}{2\hbar} (R_1-R_0)^2$}\\
    &S\equiv \text{parámetro de Huang-Rhys}\;\;\;\; (R_1-R_0)\equiv\text{diferencia entre posiciones de los mínimos}\\
    &\text{ZPL$\equiv$transición electrónica pura: $\upsilon =0\rightarrow\upsilon'\!=0$\;\;\;Su nº de onda se denomina $k_0$}\\
    &\text{Energía modo vibracional: cuantización energía vibración molécula}\equiv k_{\text{modo}}\text{ [cm$^{-1}$]}\\
&  k^{\text{abs.}}_{\text{emi.}} = k_0 \pm  m\cdot k_{\text{modo}} \;\;\;\; \text{Intensidad relativa transición: }I_{0 \rightarrow m}\!=\!e^{-S}{S^m}/{m!} \\
&\text{Evaluate: m=$0,1,...$ hasta que $I$ \!<\!<. Bar diagram $I $ vs $k_\text{abs}$ (ídem $k_\text{emi}$), width = $k_{\text{modo}}$}    \\
    &\Delta E_\text{Stokes} = k^{\text{máx}}_{\text{abs.}} -k^{\text{máx}}_{\text{emi.}}
    \begin{array}{l}
        \uparrow \Leftrightarrow (R_1\!-\!R_0)\uparrow \Leftrightarrow\text{Bandas anchas (típico metales transición)}  \\
  \downarrow \Leftrightarrow (R_1\!-\!R_0)\downarrow \Leftrightarrow\text{ZPL (zero phonon line) será la más intensa}
    \end{array}
\end{align*}

